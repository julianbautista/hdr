This manuscript summarises my research over the past 12 years in the field of observational cosmology, 
as well as the work by early career scientists for whose I was the main or the co-supervisor.  
The main topic of our work is the study of dark energy with spectroscopic galaxy data 
from the Sloan Digital Sky Survey and the Dark Energy Spectroscopic Instrument. 
We explored three quite different regimes in terms of redshift ranges and types of data:
\lya forests at high-redshift ($2 < z < 3.5$), galaxies at mid-redshift ($0.6 < z < 1.0$)
and type-Ia supernovae at low-redshift ($0 < z < 0.1$). 
We mostly focused on the statistical properties of these samples, estimating two-point functions 
and measuring the scale of baryon acoustic oscillations and the effect of redshift-space distortions. 
We explored new techniques for measuring weak-lensing from \lya forests and the link between galaxies 
and the 21cm radio signal. 
All this research is linked to the challenge of precisely measuring the expansion rate of the 
Universe and the growth-rate of structures, with the hope of finding deviations from 
the standard $\Lambda$CDM cosmological model based on general relativity. 


\vspace{0.5cm}
Keywords: cosmology, dark energy, spectroscopy
