The last 11 years of my life were dedicated to the quest of dark energy 
using data from the best spectroscopic galaxy surveys available, 
such as the Sloan Digital Sky Survey Baryon Oscillation Spectroscopic Survey (BOSS),
its extended version (eBOSS) and the Dark Energy Spectroscopic Survey (DESI).
I have been lucky to experiment with data from a diverse range of redshifts. 
In the last years of this exploration, I had the privilege to work with excellent people 
add different levels: undergrad (Andrei Marin), masters (Nattapon Preedasak), 
PhD candidates (Samantha Youles, Bastien Carreres, Vincenzo Aronica, 
Tyann Dumerchat) and postdocs (Elena Sarpa), 
as their main or co-supervisor. 
Many others were also part of it for shorter durations but equally enjoyable. 

At high-redshift, I used quasars and their \lya forests to measure baryon acoustic oscillations (BAO).
This measurement is still the highest redshift one to date, if we exclude the cosmic microwave background.  
Thanks to my experience with analysis spectra, I got involved in the details of the eBOSS data reduction 
pipeline, where my developments ultimately improved cosmological constraints.
With S. Youles, we explored a new type of systematic effect in the correlation function 
of quasars and forests, which should be further studied in future DESI data analyses. 

At mid-redshifts, I used mainly luminous red galaxies to measure their clustering 
and derive constraints from BAO and redshift-space distortions (RSD). 
Our results with eBOSS data are still the current state-of-the-art in our field, 
and will continue to be until we publish the first results with the DESI Year 1 sample. 
With T. Dumerchat and V. Aronica, we explored a joint analysis of Fourier and configuration 
space clustering for BAO and RSD. 
V. Aronica is also currently leading the effort of understanding the impact 
of photometric systematics on DESI data. 
I explored 21cm radio data from the Green Bank Telescope, and worked on its 
cross-correlation with eBOSS galaxies, which will be useful for future data from 
surveys such as the Square Kilometre Array. 

At low-redshifts, peculiar velocities can be directly measured using type-Ia supernovae. 
I turned my interest in the past two years to the joint analysis of galaxy clustering and peculiar velocities, 
in order to measure the growth-rate of structures. 
With B. Carreres, we are exploring methods to measure growth from peculiar velocity data from SNIa.
He produced the most realistic sets of mock catalogues for the Zwicky Transient Facility.
T. Dumerchat is studying emulators for a better modelling of the two-point functions in the non-linear 
regime, while E. Sarpa is testing the density-velocity method based on reconstruction techniques.

During those past years, I gathered experiences with a diverse range of cosmic epochs, 
cosmic probes, and analysis techniques. I had the opportunity of collaborating with 
a large team of researchers and more recently I supervised the work of PhD students and postdocs. 
This experience will be essential to face the challenges of the next decades, 
when experiments will push the precision of measurements to its limit. 

\section*{Future outlook}

The future in cosmology is very promising and I hope to continue providing key contributions to 
the field, as a Professor of Aix-Marseille University and the \emph{Centre de Physique des Particules de Marseille}. 

In the next four years, DESI will continue its observing program and will be 
the first stage-IV experiment to be complete, producing the best three-dimensional map 
of the Universe over $0 < z < 4$. 
The planning of its follow-up survey, DESI-II, has already started, and our team could be 
involved in its development and data analysis. The instrument would be the same but our target 
a new set of targets for scientific goals that complement those from DESI. 
The MegaMapper is a stage-V proposed experiment that would be based on a 6.5-metre telescope 
and equipped with more than 25 000 optical fibres in the focal plane. If approved, this 
project would probe the Universe over $2 < z < 5$ to learn about dark energy and inflation. 

The Euclid satellite is under construction and it is planned to be launched in 2023. 
Euclid is both a photometric and a spectroscopic survey, so it will produce weak-lensing 
and clustering measurements. The spectroscopic component will cover the range $1< z < 2$, 
with a large overlap with the DESI sample. The combination of Euclid and DESI data will 
yield the best constraints on dark energy by the end of 2030.
Lessons learned from DESI could be transferred to the analysis of Euclid data, 
and our team could be a part of this work. 

The Rubin Observatory Legacy Survey of Space and Time (Rubin-LSST) will start 
commissioning and science observations by 2024. Thanks to its cadence and depth, 
Rubin-LSST will observe light-curves for 10$^5$ supernovae over $0.1< z < 0.4$,
and produce lensing-quality images over most of the available Southern sky. 
With our expertise on the analysis of ZTF type-Ia supernovae, 
we will be well positioned to analyse the larger sample of type-Ia supernovae 
provided by Rubin-LSST. The observational program will last at least ten years
and the complementarity with spectroscopic surveys is a rich source of 
interesting projects we could lead. 


I wish to continue working towards the goal of understanding dark energy,
while forming new generations of scientists, in the role of Professor of 
Aix-Marseille University. 
I hope this manuscript convince readers of my capabilities as a researcher
and as a supervisor of a research team. 
