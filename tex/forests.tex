\chaptertoc{}

\vspace{1em}

Until today, the \lya forest in quasar spectra has been the only 
tracer of large-scale structures producing measurements of 
baryon acoustic oscillations (BAO) at redshifts above 2. 
Given the higher redshifts and the smaller scales probed, 
the clustering of the forest also yielded competitive 
constrains on the sum of the mass of the neutrino species, 
when combining it with measurements of the cosmic microwave background (CMB)
anisotropies. Current and future spectroscopic surveys plan to observe  
denser samples of \lya forests to improve our understanding of the 
$z>2$ Universe.

In this chapter, I expose my contributions to the study of
dark energy through \lya forest observations. 
Section~\ref{forests:intro} introduce the main concepts used in 
this chapter. 
Section~\ref{forests:bao} focus on the BAO measurements from 
BOSS and eBOSS surveys for which I provided key contributions.
My thesis\footnote{\url{https://tel.archives-ouvertes.fr/tel-01389967}} 
was dedicated to this topic.
Section~\ref{forests:zerrors} and \ref{forests:lensing} 
present two projects carried out by my former PhD student  
Samantha Youles at the University of Portsmouth, UK, 
who graduated on the 18th of March 2022. 


\section{Forests as a tracer of the Universe's structures}
\label{forests:intro}

The \lya forest is the name for a series of absorption lines observed in quasar
spectra caused by the presence of neutral hydrogen along the line of 
sight between us and the quasars. 
Figure \textbf{XX} shows an example of a optical quasar spectrum with its \lya forest.
These lines are seen bluewards of the \lya emission peak of the quasar,
which lies at the quasar rest-frame wavelength $\lambda_\mathrm{rest}=1216~\angstrom$.
The absorption is not limited to the first transition; bluewards of the 
Ly$\beta$ peak ($\lambda_\mathrm{rest} = 1025~\angstrom$) we also observe 
Ly$\beta$ absorption lines on top of the \lya ones, and similarly for all
Lyman series until the Lyman break at 912~$\angstrom$.

The advatage of the forests is that it provides a tomographic 
view of the matter distribution along the line of sight of the quasar. 
This is because, as the light propagates outwards of the quasar, the 
Universe expands, causing the whole spectrum to be redshifted. 
When the redshifted spectrum hits a hydrogen atom, the light being
absorbed at the \lya transition in the \emph{atom} rest-frame 
is no longer at the \lya wavelength in the \emph{quasar} rest-frame:
it a bluer wavelength. If a given quasar sits at a redshift $z_q$ and 
a given intervening atom sits at a redshift $z_a < z_q$, the \lya photon being absorbed
by the atom corresponds to a photon at $\lambda_\mathrm{rest}= 1216(1+z_a)/(1+z_q)~\angstrom$
in the quasar rest-frame. In our frame, the observed wavelength of the 
atom absorption, or equivalently of its \lya line, is 
$\lambda_\mathrm{obs}= 1216(1+z_a)~\angstrom$. Note the 
observed wavelength does not depend on the quasar redshift.
We only need the quasar redshift to identify where the \lya forest is 
in the observed spectrum, such that we can assign each observed wavelength
to a quasar rest-frame wavelength. 
The result is that a single forest of lines maps the neutral
hydrogen density accros a large range in redshifts. One single
quasar spectrum can yield a map of the large-scale distribution 
of matter along its line of sight over up to several hundreds of 
megaparsecs (Mpc). 

How much neutral hydrogen is needed to create a forest of \lya 
absorption lines? Not a lot as it turns out. Given the 
high cross-section of the \lya resonance, very low densities of neutral 
hydrogen are sufficient to create a line. 
At redshifts $2 < z < 4$ where \lya forests are observed in the optical, 
typical densities found in the intergalactic medium (IGM), $n_\mathrm{HI} \sim 10^{XXX}~\si{\cm}^{-3}$,
are enough to absorb a significant fraction of the photons. 
Denser regions completely absorb the light and create saturated lines (zero flux). 
This is the case of the so-called \emph{high-column density systems} (HCDs), 
which are often associated with galaxies or proto-galaxies. The extreme cases 
can be observed with high-resolution spectrocopy, where we can also observe a non-saturated
\emph{deuterium} absorption from which we derive constraints on big-bang nucleosynthesis (BBN, 
see section~\ref{intro:probes:bbn}). 

The \lya forest, and therefore the amounts of neutral hydrogen in the IGM, 
is tighly connected with the process of reionisation of the Universe. 
As more stars and quasars form, more ultraviolet light is produced, 
progressively ionising the neutral hydrogen. On top of that, the Universe
continuously expands, diluting it. 
Thus, the average absorption of forests decreases with time, or increases with redshift.
At redshifts above 6, most of the light bluewards of the \lya emission is absorbed, 
while at redshifts below 1, the \lya absorption is minimal. 
Forests from $z_q>6$ quasars are used to put constraints on reionisation (ref?). 

The \lya forest and its connection with the large-scale structures have been studied 
theoretically since works by \cite{gunnDensityNeutralHydrogen1965}. Most of recent 
advances are thanks to numerical simulations. In order to simulate forests, 
a full hydrodynamic n-body simulations are required, including all the 
complex baryonic physics. \textbf{Add more refs. }
On large scales, it has been shown that the forest can be considered as 
linear tracer of the underlying matter field 
\cite{mcdonaldObservedProbabilityDistribution2000, mcdonaldMeasurementCosmologicalGeometry2003, 
mcdonaldLinearTheoryPower2005}. On smaller scales, the gas physics is more 
complex to model given the effects of pressure and thermal velocities. 
Feedback from supernovae explosions, AGN or star forming galaxies also play 
an important role to model the small-scale clustering (\cite{chabanierImpactAGNFeedback2020}). 
The small scales are 
interesting due to their potential to constrain neutrino masses for instance. 


The first measurements of clustering using the \lya forests 
were based solely on the two-point statistics along the same 
line of sight (\cite{croftPowerSpectrumMass1999, mcdonaldLyAlphaForest2006}). 
Given the good wavelength sampling of a forest, 
limited by the resolution of our spectrographs, the small scales 
are easily accessible in the radial/wavelength direction. 
This type of measurement is still 
performed today and allow us to obtain tight upper limits on 
the total mass of neutrino species (\cite{palanque-delabrouilleHintsNeutrinoBounds2020}), 
when combined with measurements of the cosmic microwave background. 

It is only with the Baryon Oscillation Spectroscopic Survey (BOSS) from SDSS-III 
that we could study the clustering of \lya forests in three dimensions. 
Thanks to the density of quasars observed by BOSS, of about 15~deg$^{-2}$, 
it was possible for the first time to estimate correlations using absorption 
from neighboring lines of sight. Also thanks to the large area of sky covered by 
BOSS, the first measurement of baryon acoustic oscillations (BAO) using forests was achieved 
\cite{buscaBaryonAcousticOscillations2013, slosarMeasurementBaryonAcoustic2013, kirkbyFittingMethodsBaryon2013}.

Quasars also trace the matter field. While their density is not homogenous enough across the sky 
to measure clustering using these quasars alone, they can be cross-correlated with 
the forests. The cross-correlation between quasars and \lya forests is interesting 
because it is mostly independent of the \lya forest auto-correlation. 
This is mainly because the quasar sample has a low density (shot-noise dominated).
The first measurement of BAO in the quasar-forest cross-correlation was also possible with 
BOSS (\cite{font-riberaQuasarLymanAlphaForest2014}). The BAO constraints from 
the auto and cross correlations could be combined assuming that these two are independent. 

Since the first measurements of BAO using forests, the BOSS and eBOSS collaborations 
published BAO constraints with increasingly larger samples and improved analysis.
They are associated with the official SDSS Data Releases (DR):
\begin{itemize}
 \item DR9: First measurement of large-scale \lya correlations without BAO (\cite{slosarLymanAlphaForest2011}); 
 \item DR9: First detection of BAO in the \lya auto-correlation 
            (\cite{buscaBaryonAcousticOscillations2013, slosarMeasurementBaryonAcoustic2013, kirkbyFittingMethodsBaryon2013});
 \item DR11: First detection of BAO in the quasar-\lya cross-correlation (\cite{font-riberaQuasarLymanAlphaForest2014}), 
             updated auto-correlation measurement (\cite{delubacBaryonAcousticOscillations2015});
 \item DR12: Final BOSS auto-correlation (\cite{bautistaMeasurementBaryonAcoustic2017}) 
             and cross-correlation measurements (\cite{dumasdesbourbouxBaryonAcousticOscillations2017});
 \item DR14: Updated measurements with eBOSS data (\cite{desainteagatheBaryonAcousticOscillations2019, blomqvistBaryonAcousticOscillations2019});
 \item DR16: Final \lya BAO measurements of SDSS (\cite{dumasdesbourbouxhelionCompletedSDSSIVExtended2020}).
\end{itemize}

\section{Baryon acoustic oscillations in the forests}
\label{forests:bao}

In this section I detail the methodology I used in my past work 
to measure BAO with a set of \lya forests, highlighting my 
personal contributions. I consider the case of SDSS forests, 
which are observed in the optical domain with low-resolution spectroscopy.
Forest have a rather low signal-to-noise ratio on average, 
so the methods are fit to this type of data. 
Similar methods are employed in the measurement of the line-of-sight
(or one-dimensionnal) power-spectrum, but we focus on BAO here. 

The main steps of the BAO analysis with \lya forests are:
\begin{itemize}
    \item estimate of the transmission field and their associated weights;
    \item estimate of the two-point correlation functions, including the cross-correlation with quasars;
    \item estimate of correction matrices due to distortions of continuum fitting and metals; 
    \item fit of the BAO model over measured correlations.
\end{itemize}

\subsection{Transmission field}
\label{forests:bao:transmission}

As described in section~\ref{forests:intro}, the amount of absorbed 
flux at a given wavelength (redshift) is related to the density 
of neutral hydrogen and therefore is a tracer of structures.  

The main observable used to compute correlations is the so-called
\emph{transmission} $F$, defined as 
\begin{equation}
    F(\hat{n}, \lambda) = f(\hat{n}, \lambda)/C(\hat{n}, \lambda) = \exp{-\tau(\hat{n}, \lambda)}
    \label{eq:transmission}
\end{equation} 
where 
$f(\hat{n}, \lambda)$ is the observed flux and 
$C(\hat{n}, \lambda)$ is the unabsorbed/original flux level, 
for a quasar line-of-sight at angular position $\hat{n}$ 
and observer-frame wavelength $\lambda$, which can be linked to the
absorber redshift $z = \lambda/\lambda_\alpha - 1$ if the absorption is 
due to \lya. We can also express the transmission as a function of the 
optical depth $\tau$ as in the right-hand side of Eq.~\ref{eq:transmission}.

The challenge is to estimate transmissions from the observed fluxes of a set of quasar
spectra, particularly given that our data is low-resolution and relatively noisy.
This makes it hard to ``see'' the unabsorbed flux level $C$, also known as the 
\emph{continuum} level, requiring automated methods. Several past attemps to achive this 
used either principal-component analysis techniques 
(\cite{leeMeanfluxregulatedPrincipalComponent2012}) where the templates were built from 
high-resolution and high signal-to-noise data; or maximum likelihood methods accounting
for the non-Gaussian nature of the probability density function of $F$ 
(\cite{buscaBaryonAcousticOscillations2013}). 
During my PhD, I particularly tried to merge these 
last two methods into a single one, without success. 
Those methods, while more sophisticated and flexible in their modelling of $C$, 
they suffer from the additional noise added to the estimated transmission 
due to noisier estimates of $C$. The simplest method will prevail in the latest
analyses, which consists in averaging forests in their rest-frame to obtain a
single average shape for $C$, or \emph{mean continuum} $\bar{C}$. 
This shape is then fitted to each individual forest with a linear tilt in $\log \lambda$,
i.e., $C(\hat{n}, \lambda) \equiv \bar{C}(\lambda) [ a_0(\hat{n}) + a_1(\hat{n})\log \lambda]$,
where $a_0$ and $a_1$ are fitted coefficients per quasar. 

Once the continnum level $C$ is estimated for each forest, one can compute 
the transmission $F$ and their \emph{fluctuations} $\delta_F$ around the mean,
defined as 
\begin{equation}
\delta_F(\hat{n}, \lambda) = \frac{F(\hat{n}, \lambda)}{\langle F \rangle} - 1
\label{eq:delta_transmission}
\end{equation}
where $\langle F \rangle$ is the ensemble-averaged transmission of the Universe. 

The $\langle F \rangle$ is actually an evolving function of time (or redshift or observed-frame 
wavelength) and it is important to take this evolution into account when computing 
$\delta_F$. 
Several measurements of this quantity exist 
(\cite{faucher-giguereDirectPrecisionMeasurement2008,
parisPrincipalComponentAnalysis2011,
beckerRefinedMeasurementMean2013,
kambleMeasurementsEffectiveOptical2020}), thought the latest BAO measurements do not 
use them directly to compute $\delta_F$. One could in principle use the forests themselves
to estimate $\langle F \rangle$, by stacking forests in their observer-frame 
(by contrast with the mean continuum that is a stack in the quasar rest-frame), 
though this requires a good estimate of $C$. 
If one express $\delta_F$ as a function of 
the observed flux $f$, the continuum $C$ and the average transmission $\langle F \rangle$,
one sees that there are degeneracies. 
The latest BAO analyses therefore fits a linear function 
that models the product $C \langle F \rangle$ simultaneously. 

All methods to estimate the transmission fluctuations also suffer from the fact 
that they use information the forests themselves to estimate $C$ or $C \langle F \rangle$.
This creates spurious correlations between a given pixel in a given forest and a neighboring 
pixel in the same forest. As I will discuss in section~\ref{forests:bao:matrices},
spurious correlations are also introduced between pixels in different lines-of-sight, 
distorting the correlation function. We name this effect the \emph{distortion
by continuum fitting}; it needs to be correctly modelled when fitting the 
correlation function.

Typical ranges chosen to define the \lya forest and extract the transmission field
are between 1040 and $1200~\angstrom$ in the quasar rest-frame. Recent analyses also 
consider the Ly$\beta$ forest region, between 920 and 1020~\angstrom, 
where both \lya and Ly$\beta$ absorption are present. The absorption in this region 
can be assigned a redshift that depends on the choice of $\lambda_\mathrm{rest}$, 
that can be either \lya$ = 1216~\angstrom$  or Ly$\beta = 1025$~\angstrom. 
These rest-frame wavelengths are separated enough such that both can be used 
in clustering measurements without much contamination at separations below 200\hmpc.

Uncertainties of the transmission fluctuations are also estimated from the data 
themselves. Typically two major components are taken into account when defining 
uncertainties: instrumental noise and the instrinsic variance of absorbers.
The former is essentially the output of the data reduction pipeline, described 
in Chapter~\ref{chap:spectro}, corrected with some normalisation factor. 
The latter is estimated from the variance of $\delta_F$ observed in the data. 
The intrinsic variance of $\delta_F$ is an increasing function of redshift, 
while the instrumental noise is typically decreasing with observer-frame wavelength, 
since forest mostly lie at the blue end of the spectrographs. 
The inverse of the final pixel uncertainty squared is used as a weight in the estimates of 
the correlation function.  

Word on DLAs ? 

Word on calibration residuals and my work to find/fix the problem ? 

\subsection{Two-point correlation functions}
\label{forests:bao:correlations}

Once the transmission fluctuations $\delta_F$ and their weights $w$ are estimated
for all pixels of all forests, we compute their auto-correlation function and 
their cross-correlation with quasars (or any other point tracer) in three dimensions, 
as a function of comoving radial and transverse separations. 

The first step is to convert redshifts to comoving distances using a fiducial cosmology. 
With angular positions and distances, we can obtain the separation $\vec{r} = (\rperp, \rpara)$
between two tracers, where $\rperp$ is the component orthogonal to the line-of-sight and 
$\rpara$ is the component along the line-of-sight. 

The correlations are estimated in bins of separation. We consider usually bins of 4\hmpc\ 
from 0 to 200\hmpc in both radial and transverse directions. For the cross-correlation 
with quasars, the radial separation can be negative, meaning that the absorber is closer 
than the quasar from us. Let $A$ be the index of a separation bin, the auto-correlation is 
defined as 
\begin{equation}
\xi^\mathrm{auto}_A = \frac{\sum_{i, j ~ \mathrm{if} ~ \vec{r}_{ij} \in A} w_i w_j \delta_{F, i} \delta_{F, j}}{\sum_{i, j ~ \mathrm{if} ~ \vec{r}_{ij} \in A} w_i w_j}
\label{eq:autocorrelation} 
\end{equation}
where the indexes $i, j$ denote a given pixel in the forest. The sum is over all pairs of 
absorbers for which the separation $\vec{r}_{ij} = \vec{r}_i - \vec{r}_j$ falls inside bin $A$. 

The cross-correlation of absorbers with quasars is estimated with the following estimator: 
\begin{equation}
    \xi^\mathrm{cross}_A = \frac{\sum_{i, j ~ \mathrm{if} ~ \vec{r}_{ij} \in A} w_i w_j \delta_{F, i}}{\sum_{i, j ~ \mathrm{if} ~ \vec{r}_{ij} \in A} w_i w_j},
    \label{eq:crosscorrelation} 
\end{equation}
which is very similar to the auto-correlation one in Eq.~\ref{eq:autocorrelation}.
The index $i$ runs over absorbers while $j$ runs over quasars, but there is no $\delta_{F,j}$ term. 
This estimator is valid under a few assumptions: sparcity of quasars, weak cross-correlation 
and weak auto-correlation. These assumptions are discussed in detail in the appendix B of
\cite{font-riberaLargescaleCrosscorrelationDamped2012}. 

The covariance matrix is estimated using sub-samples of the full survey. Under the 
assumption that each sub-sample $s$ is independent, the estimator of the covariance
between $\xi_A$ and $\xi_B$ is
\begin{equation}
C_{AB} = \frac{1}{W_A W_B} \sum_{s} W_A^s W_B^s [ \xi^s_A \xi^s_B - \xi_A \xi_B],
\label{eq:covariance-subsampling}
\end{equation}
where $W_A^s$ is the total weight in bin $A$ of sub-sample $s$ and $W_A \equiv \sum_s W_A^s$. 
This estimator has been tested with 100 realisations of synthetic datasets (mocks) and 
it shows to be robust at the current precision level. 

The data vector contains around 50x50 bins, making the covariance too large (2500x2500)
for the usual number of available sub-samples ($\sim 1000$). To avoid the matrix to be 
singular, we apply a smoothing to it. The smoothing procedure considers that the 
correlation coefficients, defined as $\rho_{AB} = C_{AB}/\sqrt{C_{AA} C_{BB}}$, 
are only a function of $\Delta \rperp \equiv \rperp_B - \rperp_A$ and 
$\Delta \rpara \equiv \rpara_B - \rpara_A$. By averaging all correlation coefficients
with the same $(\Delta \rperp, \Delta \rpara)$, we obtain a 50x50 matrix which is now 
positive definite. The new covariance matrix is constructed by taking these averaged 
coefficients and multiplying them by the variances. This method is used to estimate 
covariance matrices for both auto and cross correlation functions, but also for the 
cross-covariance between $\xi^\mathrm{auto}_A$ and $\xi^\mathrm{cross}_B$, used in 
joint fits (see section~\ref{forests:bao:model}).



\subsection{Correction matrices}
\label{forests:bao:matrices}

There are two important effects to be taken into account when modelling 
the \lya correlation functions: the distortion by the continuum fitting, discussed 
in section~\ref{forests:bao:transmission}, and the contamination by metal 
absorption. The most recent analyses use matrices to convolve a binned 
cosmological model and obtain the final template to be compared to 
the measured binned correlations. 




\subsection{BAO model}
\label{forests:bao:model}




\section{Impact of redshift errors}
\label{forests:zerrors}

\section{Weak-lensing of forests}
\label{forests:lensing}



