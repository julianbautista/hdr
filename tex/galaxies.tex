\chaptertoc{}

\vspace{1em}

At redshifts between 0 and 2, galaxies can be used as tracers of
the matter distribution. With the statistics of the galaxy distribution 
we can measure characteristic scale of the baryon acoustic oscillations (BAO), 
as well as extract information about the growth-rate of structures from 
the anisotropies caused by redshift-space distortions (RSD). 
As presented in Chapter~\ref{chap:intro}, BAO and RSD are powerful 
probes of dark-energy and theories of gravity. 

In this Chapter, I overview my contributions for the study of dark-energy 
with galaxy clustering. Section~\ref{galaxies:catalogue} describes how to 
create a galaxy clustering catalogue from the spectroscopic observations, 
including and correcting for most observational systematic effects.
In section~\ref{galaxies:bao} I present the BAO measurements I performed 
with the SDSS sample of luminous red galaxies, while section~\ref{galaxies:rsd}
focus on the RSD constraints. Section~\ref{galaxies:joint} 
overviews recent work carried out by my PhD students Tyann Dumerchat 
and Vincenzo Aronica on the joint analysis of galaxy clustering in 
Fourier and configuration space. 

The work described in this chapter is published in the following 
articles: 
\cite{bautistaSDSSIVExtendedBaryon2018, 
bautistaCompletedSDSSIVExtended2020,
gil-marinCompletedSDSSIVExtended2020,
rossCompletedSDSSIVExtended2020,
zhaoCompletedSDSSIVExtended2021,
%dumerchatJoint2022
}. 

\section{From redshifts to galaxy clustering}
\label{galaxies:catalogue}

An essential step in the cosmological analysis of galaxy survey data is 
to convert convert a list of galaxy redshifts (see Chapter~\ref{chap:spectro}) 
into a catalogue from which we can define overdensities 
$\delta_n(\vec{x}) = n(\vec{x})/\bar{n} - 1$, where $n(\vec{x})$ is the 
number density 
of galaxies in a volume element located at position $\vec{x}$. 
The quantity $\bar{n}$ is the average galaxy number density over the probed 
\emph{volume}. Therefore, it is important to define precisely what is the 
volume observed, or in jargon terms, the survey window function. 

The simplest form of survey window function would be some function of position $\vec{x}$ 
that is 1 if the volume was ``observed'' and 0 else. 
One way to define this function is using a Poisson random sampling 
of the volume with points, with some arbitrary higher number density than the 
galaxy average number density (typically between 20 or 50 times higher). 
The list of points is referred to as the \emph{random} catalogue. 
In practice, the random catalogue is more complicated and accounts for 
observational completeness and systematic effects, as described below. 

In SDSS analyses, the random catalogue is a combination of an angular footprint 
and a radial distribution, both trying to matching the galaxy sample as better as possible. 

I describe now the procedure to define the angular footprint. 
The starting footprint is the same where the target selection 
(section~\ref{spectro:target})
was previously defined. Bad photometric regions or bright stars are masked out 
by removing randoms belonging to these regions. 
After tiling, fibre assignement and spectroscopic observations, the footprint 
can be divided into an unique set of sectors. Each sector corresponds to regions
observed by one or more plates. The \emph{fibre completeness} of each sector is 
defined as the ratio between the number of spectroscopically observed targets and 
the number of available targets in the sector. Random points are sub-sampled or de-weighted 
following the fibre completeness to account for it. 

Some targets cannot be observed due to 
its proximity to another observed target and the finite size of fibres, corresponding 
to 62 arcseconds in the sky. We refer to these events as \emph{fibre collisions}. 
Collisions might happen not only to pairs of galaxies, but to any group of targets with 
linking lengths smaller than 62 arcsec. Depending on the number of plates observing 
a giving sector, some collisions might be solved but the missing ones might 
impact the measured clustering on small scales if not corrected. 
While there are several methods to solve collisions 
(\cite{guoNewMethodCorrect2012, bianchiUnbiasedClusteringEstimation2017}),
we used the simplest up-weighting technique, where the nearest observed galaxy 
is up-weighted by the number of non-observed targets within the collision group. 
This assumes that non-observed targets are also galaxies of the same target type and 
that they are physically close (angularly and radially) to the observed ones.
On scales above a few \hmpc, this simple correction is a good approximation. 

Redshift failures... 

Photometric systematic weights.. 

Radial selection function is n(z)... 

Optimal weights for clustering (FKP) \dots




Reconstruction ... 

Correlation function estimation ... 

Power spectrum estimation ... 

Mock catalogues ... 

Covariance matrix ...  

\section{Baryon acoustic oscillations with galaxies}
\label{galaxies:bao}




\section{Redshift-space distortions}
\label{galaxies:rsd}

\section{Joint clustering analysis in Fourier and Configuration space}
\label{galaxies:joint}


\section{Cross-correlation with radio surveys}
\label{galaxies:radio}

