\chaptertoc{}


\vspace{1em}

In the last two decades, spectroscopy became one of the most powerful
techniques to survey the Universe. Particularly thanks to its 
capability to obtain precise redshifts, spectroscopy allows
us to build precise maps of the Universe in three dimensions. 

This chapter is an overview on how to observe galaxies with 
spectroscopy and how the data is treated all the way to the 
redshifts. I also expose my work on improving the spectroscopic
data reduction pipeline for the extended Baryon Oscillation 
Spectroscopic Survey (eBOSS). Naturally, this chapter will 
focus on the spectroscopic observations with the 
Sloan Digital Sky Survey (SDSS).


\section{Selecting the objects to observe}
\label{spectro:target}

The first step to build a spectroscopic survey is to pre-select 
objects to be observed. This step, known as \emph{target
selection}, is required since SDSS uses optical fibers to 
capture the light from an object, and therefore cannot simply
observe all objects in a given field. 

For the target selection, a prior \emph{photometric or imaging} 
survey is required. In the first N years of SDSS, a photometric
survey was carried out, covering more than 
XXX deg$^2$ of the sky \cite{yorkSloanDigitalSky2000}. 
The focal plane was equiped with six rows of five 
charge coupled device (CCD), each one covered with one of 
the SDSS filters: \textit{u, g, r, i} or \textit{z} 
\cite{gunnSloanDigitalSky1998, doiPhotometricResponseFunctions2010}.
A technique named drift-scanning was used to continuously observe 
``stripes'' of constant declination during the night.
SDSS was the first of its kind to produce a systematic survey 
of the Universe in optical.

Images were reduced using the SDSS photometric pipeline 
\cite{sdss_photo_pipe}. Fluxes/magnitudes and their uncertainties 
were computed for each detected object in five colour bands. 
Based on their fluxes and angular sizes relative to the 
point-spread function (PSF), each object received a 
photometric classification as star or galaxy (others?).

The final list of objects with their respective fluxes and 
angular positions is the input for targeting algorithms. 
These algorithms aim to select a given type of object for 
spectroscopic follow-up, based solely on their fluxes and colors. 
For galaxy surveys, it is vital to be able to distinguish between 
galaxies - the objects of our interest - and stars - which belong
to our own galaxy and have a distinct scientific purpose. 
For cosmology, since we are interested in the clustering of 
galaxies, is is also important to obtain a relatively 
homogeneous angular density of targets so to avoid spurious 
correlations. 

List of target selection papers 
(see \url{https://www.sdss.org/science/technical_publications})

\begin{itemize}
    \item SDSS Main Galaxy Sample
    \item BOSS Low-z galaxies 
    \item BOSS CMASS galaxie 
    \item BOSS \lya forest quasars \cite{nick_ross}
    \item eBOSS LRGs \cite{}
    \item eBOSS ELGs \cite{}
    \item eBOSS quasars as tracers \cite{}
    \item eBOSS \lya quasars 
\end{itemize}

\section{Pointing fibres to the sky}
\label{spectro:fibers}

Tiling, plate drilling, fibers, pluggin by hand, 
spectrographs, CCDs, observations, ...

\section{From electrons to redshifts}
\label{spectro:pipeline}

Description of automated data reduction pipeline, 
emphasis on eBOSS. 

spec2d...

spec1d...

\section{From redshifts to cosmology}
\label{spectro:catalogs}

Description of catalog generation. Shall this go on 
the clustering part ? 

